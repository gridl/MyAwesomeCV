
%%%%%%%%%%%%%%%%%%%%%%%%%%%%%%%%%%%%%%%%%
% Awesome Cover Letter
% XeLaTeX Template
% Version 1.1 (9/1/2016)
%
% This template has been downloaded from:
% http://www.LaTeXTemplates.com
%
% Original authors:
% Claud D. Park (posquit0.bj@gmail.com)
% Lars Richter (mail@ayeks.de)
% With modifications by:
% Vel (vel@latextemplates.com)
%
% License:
% CC BY-NC-SA 3.0 (http://creativecommons.org/licenses/by-nc-sa/3.0/)
%
% Important note:
% This template must be compiled with XeLaTeX, the below lines will ensure this
%!TEX TS-program = xelatex
%!TEX encoding = UTF-8 Unicode
%
%%%%%%%%%%%%%%%%%%%%%%%%%%%%%%%%%%%%%%%%%

%----------------------------------------------------------------------------------------
%	PACKAGES AND OTHER DOCUMENT CONFIGURATIONS
%----------------------------------------------------------------------------------------

\documentclass[11pt, a4paper]{awesome-cv} % A4 paper size by default, use 'letterpaper' for US letter

\geometry{left=1.5cm, top=1.5cm, right=1.5cm, bottom=2cm, footskip=.5cm} % Configure page margins with geometry
 
\fontdir[fonts/] % Specify the location of the included fonts

% Color for highlights
\colorlet{awesome}{awesome-concrete} % Default colors include: awesome-emerald, awesome-skyblue, awesome-red, awesome-pink, awesome-orange, awesome-nephritis, awesome-concrete, awesome-darknight
\definecolor{awesome}{HTML}{9370DB} % Uncomment if you would like to specify your own color

% Colors for text - uncomment and modify
%\definecolor{darktext}{HTML}{414141}
%\definecolor{text}{HTML}{414141}
%\definecolor{graytext}{HTML}{414141}
%\definecolor{lighttext}{HTML}{414141}

\renewcommand{\acvHeaderSocialSep}{\quad\textbar\quad} % If you would like to change the social information separator from a pipe (|) to something else

%----------------------------------------------------------------------------------------
%	PERSONAL INFORMATION
%	Comment any of the lines below if they are not required
%----------------------------------------------------------------------------------------

\name{Tania}{Allard}
\address{11 Central Place, Station Road, Wilmslow, SK9 1BU}
\mobile{(+44) 79 400 30 706}

\email{tania.sanchezmonroy@gmail.com}
\homepage{www.bitsandchips.me}
\github{trallard}
\linkedin{tania.sanchezmonroy}
%\skype{skypeid}
%\stackoverflow{SOid}{SOname}
\twitter{@ixek}

\position{Research Software Engineer{\enskip\cdotp\enskip} Data Science expert} % Your expertise/fields
%\position{Research Software Engineer{\enskip\cdotp\enskip}Finite Element Analysis and Computational modelling
%expert}
%\quote{``Make the change that you want to see in the world."} % A quote or statement

%----------------------------------------------------------------------------------------
%	RECIPIENT/POSITION/LETTER INFORMATION
%	All of the below lines must be filled out
%----------------------------------------------------------------------------------------

\recipient{Mozilla}
{Metal box Factory \\ Suite 441, 4th floor \\ 30 Great Guildford St \\ London SE1 0HS}

\letterdate{\today} % The date on the letter, default is the date of compilation

\lettertitle{Job Application for Senior Data Engineer} % The title of the letter

\letteropening{To whom it may concern} % How the letter is opened

\letterclosing{Yours sincerely,} % How the letter is closed

%\letterenclosure[Attached]{Curriculum Vitae} % Any enclosures with the letter

%\makecvfooter{\today}{Tania Allard~~~·~~~Cover Letter}{} % Specify the 
\makecvfooter{}{Tania Allard~~~·~~~Cover Letter}{} % Specify the letter footer with 3 arguments: (<left>, <center>, <right>), leave any of these blank if they are not needed
  
%----------------------------------------------------------------------------------------

\begin{document}

\makecvheader % Print the header

\makelettertitle % Print the title

%----------------------------------------------------------------------------------------
%	LETTER CONTENT
%----------------------------------------------------------------------------------------

\begin{cvletter}

%------------------------------------------------

%\lettersection{About Me}
I love data and innovation, which led me to pursue a PhD in computational modelling at the University of Manchester. At the moment I work as a Research Software Engineer at the University of Sheffield, where I collaborate with a number of research groups across multiple disciplines. In addition to this, I am a programming teacher/mentor, contribute to a number of open source projects including Coding Foundation (Mozilla Open Leaders) and do a significant amount of dissemination, outreach, and diversity/inclusion work (attending and speaking at conferences, local meetups, as well as writing on my personal blog www.bitsandchips.me and my work's blog www.rse.shef.ac.uk/blog).

I also work as an independent data science and  data engineer for start ups and various projects dealing with real life data across a number of business and disciplines. 

%------------------------------------------------

%\lettersection{Why this company?}
Over the course of my career I have become increasingly aware of the importance of working in the open and developing transparent, reliable, and reproducible data analysis workflows. 

\lettersection{Why Me?}
I imagine you will receive hundreds if not thousands of applications for a job like this, so I will do my best to highlight why I am the best candidate for this role:

I have been immersed in different aspects of research for the last 6 years or so. I understand the core of scientific research and have been involved in all its steps: from the early steps of experimental design and identification of the research questions, to data collection and analyses, and ultimately interpretation and dissemination of the findings. More recently, as a Research Software Engineer (which is an equivalent of an R\&E engineer) I have been involved in a range of projects consulting and supporting research teams to collect and explore large amounts of data, develop complex analysis pipelines (from data wrangling to model optimisation and validation), and better disseminate their results.  By engaging with these teams I have been able to envision adequate data collection and curation strategies, design and deploy more robust, efficient, and sustainable code and algorithms for data analysis and modelling. As a result, these teams have maximised  their research impact by making their data and code open. In addition, this has helped to add greater value to their digital assets (code, data, preprints, journals, visualization).

The role of an RSE or Research and experiments engineer is fundamentally different from a traditional software engineering role since it requires not only programming and software development skills but also, a deep understanding of the scientific method as well as being able to communicate complex scientific across teams efficiently. We basically work at the intersection of pure science (including data science and machine learning) and software engineering, or better acting as a bridge between these two.  As such, I have also become an advocate for best practices and reproducibility within all the projects and teams I collaborate, ensuring that the results drawn are of the highest quality possible and follow scientific rigour. 

You will be able to see on my GitHub profile a number of projects I work on as well as the various organisations I contribute to. And this is one of the things I like the most of this kin  of roles: the versatility of projects and teams one gets to work with: from hard sciences to humanities and across scopes: from data science and machine learning to web development. 

I hope this letter has served to highlight why I would be a perfect match for this role and how passionate I am about what I do and what is that I can bring to the team. I think this is a very exciting opportunity and I would be able to bring a number of skills and contribute to Mozilla's efforts on open innovation, reproducible science, and internet health.  I  would very much appreciate if I were able to discuss this opportunity further and how I could be an asset for the team.
%------------------------------------------------

\end{cvletter}

%----------------------------------------------------------------------------------------

\makeletterclosing % Print the signature and enclosures

\end{document}